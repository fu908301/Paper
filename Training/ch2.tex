% File name: ChapterForm.tex
%
% Author: your's name
%
% Creation: 2002/06/25
%
% !!!: The chapter form only can be used with the "thesis" document style!!!
\chapter{Related Work}
\vspace*{30pt} Some applications require that periodic patterns must
satisfy two extra conditions including \emph{Min\_count} and
\emph{Max\_distance} \cite{Anwar2015}. Parameter \emph{Min\_count}
is used to limit that a pattern needs to repeat itself at least a
certain number of times to demonstrate its significance and
periodicity. Parameter \emph{Max\_distance} is to limit that the
distance between the same two patterns has to be within some
reasonable bound \cite{AA2004, Patnaik2008, Coan2003}.

\section{Ch2 Ch2}
Usually, \emph{Min\_count} is set to be a value greater than or
equal to 2, and \emph{Max\_distance} is set to be \emph{N}/2, where
\emph{N} is the length of the time series data. To provide accurate
period patterns for some applications, the length of the period
pattern may be limited \cite{Pippa2016}. Moreover, to provide
efficient processing, some pruning techniques are preferred
\cite{Patnaik2008}.

\subsection{Related}
For the example of the time series data \emph{abcabdabe}, both
patterns \emph{ab} and a start in positions 0, 3, 6. In this case,
we usually prefer only a pattern \emph{ab}. That is, a pattern
\emph{a} is pruned since it is the subset of a patter \emph{ab} in
the same starting position. Similarly, a pattern \emph{b} is pruned
since a pattern \emph{b} is the subset of patter \emph{ab} in the
same ending positions.

\subsubsection{Work}
In the past of the structure, the suffix tree is a compact version
of the suffix tries \cite{Pablo2006}. A suffix tree of length
\emph{m} is a tree with the following properties: (1) Each tree edge
is labelled by a substring of \emph{S}. (2) Each internal node has
at least 2 children. (3) The number of leaves is m. (4) Each suffix
has its unique leaf. Furthermore, it is well established in exact
string matching and a good introduction to use in biology (for the
example, in constructing and searching the DNA sequence)
\cite{Patnaik2008}.

\section{Related Work}
A suffix tree indexes a string of length \emph{N} which means that
it has \emph{N} leaf nodes. When the size of the sequence database
increases, the storage space of a suffix tree also increases. To
reduce the storage space, the suffix array is proposed which is
basically a sorting list of all the suffixes of strings and only the
sorting list is stored \cite{AT2009}. The main advantage of the
suffix array over the suffix tree is that the suffix array uses
three to five times less space.
