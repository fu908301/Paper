\small
\setlength{\baselineskip}{12pt}
\bibliographystyle{unsrt}
\begin{thebibliography}{99}

\bibitem{bene}
R.~Benetis, C.~S.~Jensen, G.~Karciauskas, and S.~Saltenis,
``Nearest Neighbor and Reverse Nearest Neighbor Queries for Moving Objects,''
{\em http://www.cs.auc.dk/tbp/Teaching/DAT5E01/benetis.pdf,} pp.~1--18, 2001.

\bibitem{bent75}
J.~L.~Bentley,
``Multidimensional Binary Search Trees Used for Associative Searching,''
{\em Communications of the ACM,}
Vol.~18, No.~9, pp.~509--517, Sept.~1975.

\bibitem{bent79}
J.~L.~Bentley and J.~H.~Friedman,
``Data Structure for Range Search,''
{\em ACM Computing Surveys,}
Vol.~11, No.~4, pp.~397--409, Dec.~1979.

\bibitem{brin}
T.~Brinkhoff, H.~Kriegel, and B.~Seeger,
``Multi-Step Processing of Spatial Joins,''
{\em Proc.~of ACM SIGMOD Int.~Conf.~on Management of Data},
pp.~197--208, 1994.

\bibitem{gunt}
O.~Gunther and J.~Bilmes,
``Tree-Based Access Methods for Spatial Databases: Implementation and Performance Evaluation,''
{\em IEEE Trans.~on Knowledge and Data Eng.,}
Vol.~3, No.~3, pp.~342--356, Sept.~1991.

\bibitem{kuma}
A.~Kumar,
``G-Tree: A New Data Structure for Organizing Multidimensional Data,''
{\em IEEE Trans.~on Knowledge and Data Eng.,}
Vol.~6, No.~2, pp.~341--347, April 1994.

\bibitem{leut}
S.~T.~Leutenegger and M.~A.~Lopez,
``The Effect of Buffering on the Performance of R-Trees,''
{\em Proc.~of the 14th Int.~Conf.~on Data Engineering,}
pp. 164--171, 1998.

\bibitem{lome}
D.~B.~Lomet and B.~Salzberg,
``The hB-Tree: A Multiattribute Indexing Method with Good Guaranteed Performance,''
{\em ACM Trans.~on Database Systems,}
Vol.~15, No.~4, pp.~625--658, Dec.~1990.

\bibitem{same}
H.~Samet,
{\em The Design and Analysis of Spatial Data Structures,}
Addison Wesley, Reading, MA, 1990.

\bibitem{tung}
C.~D.~Tung, W.~C.~Hou, and J.~H.~Chu,
``Multi-Priority Tree: An Index Structure for Spatial Data,''
{\em Proc.~of Int.~Computer Symposium,}
pp.~1285--1290, 1994.

\bibitem{yeh01}
W.~H.~Yeh,
``A Hybrid Approach-Based Signature Extraction Method for Similarity Retrieval,����
{\em Master Thesis,}
Dept. of Computer Science and Eng., National Sun Yat-Sen University, June 2001.

\end{thebibliography}
