% File name: ChapterForm.tex
%
% Author: your's name
%
% Creation: 2002/06/25
%
% !!!: The chapter form only can be used with the "thesis" document style!!!
\chapter{Method}
\vspace*{30pt} In this section, we present the time-position join
(\emph{TPJ}) method which improves the suffix tree method in the
first phase of Rasheed \emph{et al}.'s approach.

\section{Ch3 Ch3}
In Phase 1 of Rasheed \emph{et al}.'s approach for periodicity
mining in time series databases, it uses a revised version of the
suffix tree as the data structure to generate candidate patterns.

\section{Method}
For the input \emph{abcabbabb\$} as shown in Figure
\ref{licesuffixtree}, it needs to construct a suffix tree of 14
nodes as shown in Figure \ref{licesuffixtreebottumup}. However, in
fact, patterns \emph{ab}(0, 3, 6), \emph{abb}(3, 6), \emph{b}(1, 4,
5, 7, 8), \emph{bb}(4, 7) are candidate patterns.

\begin{figure}
\begin{center}
\centerline{\psfig{figure=fp-tree.eps,width=10cm}}
%\vspace{4cm}
\caption{The suffix tree for the string \emph{abcabbabb\$}}
\label{licesuffixtree}
\end{center}
\end{figure}

\begin{figure}
\begin{center}
\centerline{\psfig{figure=licesuffixtreebottumup.eps,width=10cm}}
%\vspace{4cm}
\caption{Suffix tree for string \emph{abcabbabb\$} after bottum-up
traversal} \label{licesuffixtreebottumup}
\end{center}
\end{figure}
